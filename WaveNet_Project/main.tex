\documentclass[a4paper, 11pt]{article}

\usepackage[utf8]{inputenc}
\usepackage{amsmath, amssymb}
\usepackage{geometry}
\geometry{top=2cm, bottom=2cm, left=2cm, right=2cm}
\usepackage{graphicx}
\usepackage{hyperref}
\usepackage{fancyhdr}
\usepackage{lipsum}

% Sans-serif font
%\renewcommand\familydefault{\sfdefault}
\usepackage{mathptmx}



\title{WaveNet Project: ICTP Post Doc}
\author{Mahbub Rahaman}
\date{\today}

\begin{document}

\maketitle

\section{Introduction}

WaveNets aims to establish a novel theoretical paradigm for understanding quantum systems, centred on a network interpretation of many-body wave-functions. Ongoing experimental progress motivates the need for a new theoretical approach: in the field of quantum simulation and quantum computing, probing capabilities have reached unprecedented levels, with the ability to collect thousands of wave function snapshots with impressive accuracy. However, most of our theoretical understanding of such settings still relies on and relates to few-body observables. This has created a clear gap between experimental capabilities and theoretical tools and concepts available to understand physical phenomena. 

The overall goal of WaveNets is to bridge this gap by introducing a mathematical framework to describe wave-function snapshots based on network theory — wave function networks —  that will enable a completely new set of tools to address open problems in the field of quantum matter.

\section{Goal}
WaveNets' main objectives are:

\begin{itemize}
    \item Demonstrate that wave function snapshots of correlated systems are described by scale-free networks, and classify the robustness of quantum simulators according to such;
    \item Formulate methods for quantifying the Kolmogorov complexity of many-body systems, and propose an information-theory-based characterization of topological matter and confinement in gauge theories;
    \item Propose scalable methods for measuring entanglement in quantum simulators and computers, as well as for their validation.
\end{itemize}
Achieving these objectives will enable us to:
\begin{itemize}
    \item Provide unique insights into the information structure of quantum matter.
    \item Enable methods of probing and controlling matter of direct experimental relevance, thanks to the intrinsic scalability of network-type descriptions.
    \item Establish a new, interdisciplinary bridge between quantum science and network/data mining theory, enabling knowledge transfer between two mature, yet poorly connected disciplines.
\end{itemize}

\section{Theoretical Background}
\subsection{Network Theory}
Network theory is a branch of mathematics and computer science that studies the structure and dynamics of networks, which are collections of nodes (or vertices) connected by edges (or links). Networks can represent a wide range of systems, from social networks to biological systems, and even quantum systems. The key concepts in network theory include:
\begin{itemize}
    \item \textbf{Nodes}: The individual entities in a network, which can represent particles, individuals, or quantum states.
    \item \textbf{Edges}: The connections between nodes, which can represent interactions, correlations, or relationships.
    \item \textbf{Degree}: The number of edges connected to a node, which can indicate the node's importance or influence within the network.
    \item \textbf{Path}: A sequence of edges that connects two nodes, which can represent a route or a flow of information.
    \item \textbf{Network Topology}: The arrangement of nodes and edges in a network, which can affect the network's properties and behavior.
\end{itemize}



WaveNets is based on the idea that many-body wave functions can be interpreted as networks, where nodes represent quantum states and edges represent correlations between them. This perspective allows us to apply tools from network theory to analyze quantum systems, leading to new insights and methods for understanding complex quantum phenomena.


\end{document}