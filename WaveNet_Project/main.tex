\documentclass[a4paper, 11pt]{article}

\usepackage[utf8]{inputenc}
\usepackage{amsmath, amssymb}
\usepackage{geometry}
\geometry{top=2cm, bottom=2cm, left=2cm, right=2cm}
\usepackage{graphicx}
\usepackage{hyperref}
\usepackage{fancyhdr}
\usepackage{lipsum}
\usepackage{color}
\usepackage{physics}

% Sans-serif font
%\renewcommand\familydefault{\sfdefault}
\usepackage{mathptmx}



\title{WaveNet Project: ICTP Post Doc}
\author{Mahbub Rahaman}
\date{\today}

\begin{document}

\maketitle

\section{Introduction}

WaveNets aims to establish a novel theoretical paradigm for understanding quantum systems, centred on a network interpretation of many-body wave-functions. Ongoing experimental progress motivates the need for a new theoretical approach: in the field of quantum simulation and quantum computing, probing capabilities have reached unprecedented levels, with the ability to collect thousands of wave function snapshots with impressive accuracy. However, most of our theoretical understanding of such settings still relies on and relates to few-body observables. This has created a clear gap between experimental capabilities and theoretical tools and concepts available to understand physical phenomena. 

The overall goal of WaveNets is to bridge this gap by introducing a mathematical framework to describe wave-function snapshots based on network theory — wave function networks —  that will enable a completely new set of tools to address open problems in the field of quantum matter.

\section{Goal}
WaveNets' main objectives are:

\begin{itemize}
    \item Demonstrate that wave function snapshots of correlated systems are described by scale-free networks, and classify the robustness of quantum simulators according to such;
    \item Formulate methods for quantifying the Kolmogorov complexity of many-body systems, and propose an information-theory-based characterization of topological matter and confinement in gauge theories;
    \item Propose scalable methods for measuring entanglement in quantum simulators and computers, as well as for their validation.
\end{itemize}
Achieving these objectives will enable us to:
\begin{itemize}
    \item Provide unique insights into the information structure of quantum matter.
    \item Enable methods of probing and controlling matter of direct experimental relevance, thanks to the intrinsic scalability of network-type descriptions.
    \item Establish a new, interdisciplinary bridge between quantum science and network/data mining theory, enabling knowledge transfer between two mature, yet poorly connected disciplines.
\end{itemize}

\section{Theoretical Background}
\subsection{Network Theory}
Network theory is a branch of mathematics and computer science that studies the structure and dynamics of networks, which are collections of nodes (or vertices) connected by edges (or links). Networks can represent a wide range of systems, from social networks to biological systems, and even quantum systems. The key concepts in network theory include:
\begin{itemize}
    \item \textbf{Nodes}: The individual entities in a network, which can represent particles, individuals, or quantum states.
    \item \textbf{Edges}: The connections between nodes, which can represent interactions, correlations, or relationships.
    \item \textbf{Degree}: The number of edges connected to a node, which can indicate the node's importance or influence within the network.
    \item \textbf{Path}: A sequence of edges that connects two nodes, which can represent a route or a flow of information.
    \item \textbf{Network Topology}: The arrangement of nodes and edges in a network, which can affect the network's properties and behavior.
\end{itemize}

\subsubsection{Scale-Free Network}
A \textbf{scale-free network} is a type of network whose degree distribution follows a power law, at least asymptotically. In such networks, most nodes have only a few connections, while a small number of nodes, known as hubs, have a very large number of connections. This property leads to several distinctive features:

\begin{itemize}
    \item \textbf{Power-law degree distribution}: In scale-free networks, the degree distribution $P(k)$—the probability that a randomly chosen node has $k$ connections—follows a power law: $P(k) \sim k^{-\gamma}$, where $\gamma$ is a positive constant typically between 2 and 3. This means that while most nodes have only a few connections, there is a non-negligible probability of finding nodes with very high degree. The power-law behavior is a hallmark of scale-free networks and distinguishes them from random or regular networks, which typically have a Poisson or narrow degree distribution.
    \item \textbf{Emergence of hubs}: The power-law distribution leads to the formation of hubs—nodes with a significantly higher number of connections than average. These hubs play a crucial role in the network's structure and dynamics, often acting as central points for communication, transport, or correlation. In physical or quantum systems, hubs may correspond to highly entangled states or configurations that dominate the system's behavior.
    \item \textbf{Robustness and vulnerability}: Scale-free networks exhibit remarkable robustness against random failures; removing a random node is unlikely to disrupt the overall connectivity, since most nodes are sparsely connected. However, the network is highly vulnerable to targeted attacks on hubs. Removing a few key hubs can fragment the network and drastically reduce its functionality. In quantum systems, this translates to resilience against random noise but sensitivity to perturbations affecting highly correlated states.
    \item \textbf{Self-similarity}: Scale-free networks often display self-similarity or fractal-like properties. This means that the structure of the network at different scales is similar—the pattern of hubs and connections repeats across sub-networks. Self-similarity is important for understanding hierarchical organization and scaling behavior in complex systems, including quantum many-body states.
\end{itemize}

\paragraph{Formation Mechanisms}
The most common mechanism for the emergence of scale-free networks is \textit{preferential attachment}, as described by the Barabási–Albert (BA) model. In this model, new nodes are more likely to connect to existing nodes with higher degrees, leading to the rich-get-richer phenomenon and the formation of hubs.

\paragraph{Applications}
Scale-free networks are found in many real-world systems, including the internet, social networks, biological networks (such as protein interaction networks), and citation networks. Their properties have profound implications for the dynamics, resilience, and evolution of complex systems.

\subsubsection{Scale-Free Networks in Quantum Mechanics}

In quantum mechanics, especially in the study of many-body systems, the concept of scale-free networks can be applied to the structure of wave function correlations. Here, nodes may represent quantum states, basis configurations, or subsystems, while edges encode the strength or presence of quantum correlations (such as entanglement or transition amplitudes).

Recent research suggests that the network formed by the correlations in many-body wave functions often exhibits scale-free properties. This means that a few quantum states (or configurations) act as hubs, being highly correlated with many others, while most states are only weakly connected. Such a structure reflects the hierarchical organization of quantum correlations and can be linked to phenomena such as quantum criticality, topological order, and robustness against decoherence.

\paragraph{Implications}
\begin{itemize}
    \item \textbf{Robustness}: Quantum systems with scale-free correlation networks may be resilient to random noise or perturbations, as the overall connectivity is maintained by the hubs.
    \item \textbf{Entanglement structure}: The presence of hubs can indicate regions of high entanglement, which are crucial for quantum information processing and simulation.
    \item \textbf{Complexity and classification}: Analyzing the scale-free nature of quantum networks provides new tools for classifying phases of matter, understanding quantum chaos, and probing the complexity of quantum states.
\end{itemize}

By leveraging network theory and the concept of scale-free networks, WaveNets aims to provide a powerful framework for analyzing and understanding the intricate structure of quantum many-body systems, bridging the gap between experimental data and theoretical models.




In the context of quantum systems, a scale-free network refers to a network where the degree distribution of nodes—representing quantum states or configurations—follows a power-law. This means that most quantum states are weakly connected, while a few states act as hubs with many correlations to others. Such a structure can emerge in many-body wave function networks, reflecting the underlying complexity and hierarchy of correlations in quantum matter. 

The presence of scale-free properties implies robustness in the quantum system against random perturbations, but also sensitivity to changes affecting highly connected states. Understanding and identifying scale-free behavior in wave function networks is central to the WaveNets approach, as it provides a framework for classifying and probing the resilience and structure of quantum simulators.

\section{Wave Function Networks}
WaveNets is based on the idea that many-body wave functions can be interpreted as networks, where nodes represent quantum states and edges represent correlations between them. This perspective allows us to apply tools from network theory to analyze quantum systems, leading to new insights and methods for understanding complex quantum phenomena.


\end{document}