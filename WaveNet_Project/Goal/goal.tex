\documentclass[a4paper, 11pt]{article}

\usepackage[utf8]{inputenc}
\usepackage{amsmath, amssymb}
\usepackage{geometry}
\geometry{top=2cm, bottom=2cm, left=2cm, right=2cm}
\usepackage{graphicx}
\usepackage{hyperref}
\usepackage{fancyhdr}
\usepackage{lipsum}
\usepackage{color}
\usepackage{physics}

% Sans-serif font
%\renewcommand\familydefault{\sfdefault}
\usepackage{mathptmx}

% Blue and red colors newcommand
\usepackage{xcolor}
\definecolor{blue}{RGB}{0, 0, 255}
\definecolor{red}{RGB}{255, 0, 0}

\title{Proposed Research Goal for the ICTP Condensed Matter \& Statistical Physics Postdoctoral Position 2026 - ERC WaveNets Project}
\author{Dr. Mahbub Rahhaman}
\date{} % Removes the date from the title

\begin{document}

\maketitle

\section*{1. Introduction and Project Alignment}
I am highly enthusiastic about the ICTP Condensed Matter \& Statistical Physics Postdoctoral Position 2026 associated with the ERC WaveNets project. My research interests align perfectly with the project's ambitious goals of unraveling the intricate dynamics of wave propagation in disordered and active media, and leveraging these insights for novel functionalities. I propose to investigate \textbf{[Your specific research area, e.g., "the role of non-Hermitian degeneracies in shaping wave transport in complex, engineered networks"]} within the context of WaveNets, building upon my expertise in \textbf{[Your relevant expertise, e.g., "theoretical condensed matter physics and numerical simulations of wave phenomena"]}.

\section*{2. Proposed Research Questions/Objectives}
My primary research objective during this fellowship will be to explore the emergence of robust and tunable wave transport phenomena in complex network architectures, specifically focusing on the interplay between network topology, disorder, and non-Hermitian effects. This will involve addressing the following key questions:
\begin{itemize}
    \item How do specific topological features and local symmetries in complex networks influence the localization and delocalization of waves, particularly in the presence of disorder and non-Hermitian elements?
    \item Can we develop theoretical models to predict and control the flow of waves in dynamic or active WaveNets, considering the effects of feedback and non-equilibrium driving?
    \item What are the universal statistical properties of wave fields in large-scale, disordered WaveNets, and how can these properties be harnessed for sensing or information processing applications?
\end{itemize}

\section*{3. Methodology and Approach}
To address these questions, I plan to employ a combination of advanced analytical techniques and state-of-the-art numerical simulations. Analytically, I will utilize approaches from random matrix theory, effective medium theories, and Green's function formalism to understand the spectral properties and transport characteristics of waves in complex networks. Computationally, I will develop and utilize large-scale simulation frameworks, potentially employing techniques such as finite-difference time-domain (FDTD) methods, transfer matrix methods, and network propagation algorithms. I will leverage computational resources to explore diverse network topologies and parameter spaces relevant to both theoretical predictions and potential experimental realizations.

\section*{4. Expected Outcomes and Broader Impact}
The expected outcomes of this research include a comprehensive theoretical understanding of robust wave transport in complex, non-Hermitian networks, the identification of novel design principles for controlling wave propagation, and the development of computational tools for simulating large-scale WaveNets. This work will directly contribute to the ERC WaveNets project's aim of advancing fundamental understanding of wave phenomena in complex systems and could inspire new directions in areas such as topological photonics, acoustic metamaterials, and quantum information processing.

\section*{5. Synergy with Professors XX and YY}
I am particularly keen to work with \textbf{Professor XX} and \textbf{Professor YY} during this fellowship. Professor XX's pioneering work on \textbf{[mention specific area, e.g., 'non-Hermitian topological phases in disordered systems' or 'the theory of wave localization in random media']} is directly relevant to my proposed research on robust wave phenomena in complex networks. I believe their profound expertise in \textbf{[specific technique/theory, e.g., 'the analytical treatment of non-Hermitian Hamiltonians' or 'the application of random matrix theory to wave transport']} will be invaluable in developing the theoretical models necessary to describe the intricate interplay of disorder and non-Hermiticity in WaveNets.

Furthermore, Professor YY's extensive contributions to \textbf{[mention specific area, e.g., 'statistical mechanics of complex networks' or 'the development of advanced numerical methods for wave propagation simulations']} perfectly complement my computational approach. I am eager to learn from their experience in \textbf{[specific skill, e.g., 'designing efficient algorithms for large-scale network simulations' or 'analyzing the statistical properties of complex wave fields']}, which will be crucial for validating my theoretical predictions and exploring realistic network configurations. I am confident that a collaborative environment with both Professor XX and Professor YY will significantly enrich my postdoctoral experience and accelerate the progress of my research within the WaveNets project.

\section*{6. Long-Term Vision}
This postdoctoral position at ICTP, under the umbrella of the ERC WaveNets project and with the mentorship of Professors XX and YY, represents an ideal opportunity to deepen my expertise in cutting-edge condensed matter physics and statistical mechanics. This experience will be instrumental in developing my independent research profile and preparing me for a successful career in academia.

\end{document}


\end{document}